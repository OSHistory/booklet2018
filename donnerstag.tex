\renewcommand{\conferenceDay}{\donnerstag}
\newsmalltimeslot{09:00}

\abstractAPH{Otto Daussau}%
{GBD Web Suite}{}%
{
In diesem Vortrag wird die neue GBD Web Suite vorgestellt mit der Möglichkeit, Daten aus externen
Fachanwendungen sowie mit QGIS aufbereitete Projekte zu integrieren und über die Komponenten GBD
Web Server und GBD WebGIS Client darzustellen.%
}

% time: 2018-03-22 09:00:00
\abstractZwei{Jörg Thomsen}%
{Wie kommt der Schwimmbagger ins Web-GIS?}%
{}%
{%
Am Beispiel eines Unternehmens aus der Rohstoffbranche wird eine vollkommen automatisierten
Datenverarbeitungskette aufgezeigt.

Ein Schwimmbagger nimmt, während er über den Baggersee schwimmt
und baggert, kontinuierlich seine Geoposition auf und misst gleichzeitig zu jeder Geokoordinate die
Wassertiefe. Diese Daten sollen in einem Web-GIS dargestellt und wöchentlich aktualisiert werden.
Wenn es aber nicht nur einen Bagger auf einem See gibt, sondern viele Bagger auf vielen Seen, ist
eine automatische Verarbeitung der Daten bis hin zum WMS gefordert.%
}


% time: 2018-03-22 09:00:00
\abstractVier{Andreas Krumtung}%
{Potenziale und Herausforderungen eines Open-Innovation-Ansatzes für offene Geo- und
Vermessungsdaten der öffentlichen Verwaltung}%
{}%
{%
% originale Kurzbeschreibung fing mitten im Satz. Daher vorn vervollständigt und hinten
%  abgeschnitten.
%TODO Wenn Werbung dahinter passt, kurz lassen, sonst Weißraum füllen.
Spätestens mit dem Beitritt Deutschlands zur Open Government Partnership im Dezember 2016 und der
Veröffentlichung des ersten nationalen Aktionsplans im Sommer 2017 hat das Thema Open Government
innerhalb der Verwaltungen Deutschlands große Bedeutung erlangt. Als Grundlage eines offenen
Regierungs- und Verwaltungshandelns gelten vor allem offene Daten. Der Bund und einige
Bundesländer haben dementsprechende Open"=Data-, Transparenz- oder Informationsfreiheitsgesetze
verabschiedet, \dots%
}

\newtimeslot{09:35}
% time: 2018-03-22 09:35:00
\abstractAPH{}%
{Lightning Talks}%
{}%
{%
  \vspace{-2em}
  \begin{itemize}
    \item \emph{Peter Lanz}: Katastrophenhilfe für die zivile Seenotrettung im Mittelmeer
  \end{itemize}

  \noindent Für weitere Lightning Talks in diesem Vortragsslot beachten Sie bitte siehe die Aushänge.
}

% time: 2018-03-22 09:35:00
\abstractZwei{Thomas Schüttenberg}%
{Jetzt in Ihrem QGIS: ISYBAU XML-Abwasserdaten\vspace{0.2em}}%
{In der Hauptrolle der OGR GMLAS Treiber}%
{%
Dies ist ein Arbeitsbericht über die Verwendung des OGR-GMLAS-Treibers für die Nutzung der
\emph{ISYBAU"=Austauschformate Abwasser (XML)} in QGIS~3.

Ziel ist einerseits die Anzeige von Kanälen,
Schächten und Abwasserbauwerken (möglichst) auf Knopfdruck, andererseits eine freie
ISYBAU"=Schnittstelle zu anderen abwasserbezogenen QGIS-Projekten und Werkzeugen, wie z.\,B. den
QKan-Plugins oder der schweizerischen Abwasserfachschale QGEP, die auf diese Weise auch für Anwender
aus Deutschland interessant werden könnte.%
}
\newpage

% time: 2018-03-22 09:35:00
\abstractVier{Christian Strobl}%
{CODE-DE -- der nationale Zugang zu Copernicus-Daten für Deutschland}%
{}%
{%
% Beschreibung am Ende vervollständigt, war einfach abgeschnitten.
%TODO Länge der Beschreibung ändern, wenn Platz für Werbung erforderlich.
Die \emph{Copernicus Data and Exploitation Platform -- Deutschland} (CODE-DE) ist der nationale
Copernicus-Zugang für die Satellitendaten der Sentinel-Satellitenreihe und die
Informationsprodukte der Copernicus-Dienste. CODE-DE wird speziell Nutzern in Deutschland -- von
Behörden über Forschungseinrichtungen und Unternehmen bis hin zu Privatpersonen -- einen einfachen
und schnellen Zugang zu den Daten und Informationen aller operationellen Sentinel"=Satelliten sowie
der Copernicus-Dienste~\dots %
}


\newtimeslot{10:10}
% time: 2018-03-22 10:10:00
\abstractAPH{Arne Schubert}%
{YAGA}%
{Yet Another Geo Application}%
{%
Das YAGA-Development-Team stellt seinen finalen Release von leaflet-ng2~1.0.0, einer ganularen
Integration von Leaflet in Angular~2 und folgende Versionen, vor. Es werden Vorteile und Modularität
des Frameworks herausgestellt. Zudem werden weitere Module und die künftige Roadmap rund um das
Framework vorgestellt.%
}

% time: 2018-03-22 10:10:00
\abstractZwei{Jörg Höttges}%
{QKan -- QGIS-Plugins zur \mbox{Aufbereitung} von Kanalnetzdaten\linebreak für \mbox{Simulationen}\vspace{0.2em}}%
{Aktueller Stand und weitere Ziele}%
{%
QKan ist ein System aus QGIS-basierten Plugins, das zur Vor- und Nachbereitung von Daten zu
kommunalen Entwässerungssystemen im Zusammenhang mit hydrodynamischen Simulationen dient. Die Daten
werden in einer SpatiaLite-Datenbank gespeichert und können sowohl mit Hilfe der Plugins als auch
mit den QGIS-Funktionen verarbeitet werden. Es werden der aktuelle Stand sowie die nächsten
geplanten Entwicklungsschritte vorgestellt.%
}

% time: 2018-03-22 10:10:00
\abstractVier{Thomas Eiling}%
{Refaktorieren oder grüne Wiese?}%
{Die Reise von opencaching.de von einer Legacy-Applikation zu Symfony Full Stack mit \mbox{responsivem} Webdesign.}%
{%
Seit dem März 2016 entwickelt das Team von Openaching Deutschland an einer neuen Webseitenversion
auf Basis von Symfony und Bootstrap. In diesem Vortrag möchten wir euch mit auf die Reise nehmen,
über unsere erreichten Ziele und genommen Hürden und Hindernisse berichten, die eine so große
Legacy-Applikation mit sich bringt.%
}


\newtimeslot{11:05}
% time: 2018-03-22 11:05:00
\abstractAPH{Katrin Hannemann}%
{Erstellung individueller Symbole in Inkscape für die Verwendung in QGIS}%
{}%
{%
Dieser Vortrag gibt einen Überblick über die Erstellung individueller SVG-Marker in Inkscape und
deren Verwendung in QGIS. Dabei werden zunächst vorhandene Möglichkeiten in QGIS gezeigt. In
Inkscape werden Benutzeroberfläche und wichtige Werkzeuge vorgestellt und erklärt, wie individuelle
Symbole erzeugt werden können. Anschließend wird gezeigt, wie die individuell erstellten Symbole in
QGIS verwendet und entsprechend angepasst werden können.  Abschließend wird gezeigt, welche
Möglichkeiten es gibt, die neuen Symbole bereitzustellen und  mit der~\dots%
}

% time: 2018-03-22 11:05:00
\abstractZwei{Stefan Kuethe}%
{Dockerize stuff}%
{}%
{%
Dieser Talk zeigt wie man mit Docker und Docker Swarm in wenigen Schritten ein PostGIS-Cluster mit
einem Manager und mehreren Workern aufsetzt und mit einem GeoServer"=Container verbindet. Darüber
hinaus werden weitere Geo-Container (OSRM, Mapnik, MapShaper,~\dots) kurz vorgestellt.%
}

% time: 2018-03-22 11:05:00
\abstractVier{Hartmut Holzgraefe}%
{OSM-Daten zu Papier bringen}%
{}%
{%
Es gibt viele Online-Dienste, die auf Basis von OSM-Daten schöne Karten generieren, aber nur sehr
wenige davon eignen sich auch für Ausdrucke auf Papier. Mit MapOSMatic existiert eine
Online-Lösung, die diese Lücke schließen will.%
}

\newtimeslot{11:40}
% time: 2018-03-22 11:40:00
\abstractAPH{Peter Gipper}%
{Entwicklung von Plug-Ins für QGIS 3 -- eine Einführung}%
{}%
{%
Dieser Vortrag widmet sich der Entwicklung von Plugins für QGIS~3 und ist vor allem an Entwickler
oder Hobbyprogrammierer gerichtet, die bereits Plug-Ins für QGIS~2 programmiert haben. Die
Entwicklung für eine Software, die noch in Entwicklung ist bzw. noch nicht etabliert ist, bringt
einige Hürden mit sich und führt zu vielen Fragen. Dieser Vortrag geht speziell auf Änderungen ein,
die beim Umstieg von QGIS~2 auf QGIS~3 relevant werden.%
}

% time: 2018-03-22 11:40:00
\abstractZwei{Volker Mische}%
{Noise}%
{Einfach Daten durchsuchen}%
{%
  Noise ist eine neue Bibliothek, die dazu dient Daten im JSON-Format zu durchsuchen. Eine einfache
  Handhabung, sowohl bei der Administration als auch bei der Datenabfrage ist zentrales Ziel. Der
  Vortrag gibt einen Überblick über die verwendeten Technologien Rust und RocksDB und mündet in eine
  Live-Demonstration, die u.\,a. die intuitive Abfragesprache vorstellt. Noise ist Open Source unter
  Apache~2.0-/MIT-Lizenz.%
}

% time: 2018-03-22 11:40:00
\abstractVier{Tobias Knerr}%
{3D Model Repository~-- von der \mbox{Parkbank} bis zur Burg}%
{Freie 3D-Modelle für OpenStreetMap}%
{%
In OpenStreetMap werden zunehmend komplexe 3D"=Modelle erstellt. Mit deren Detailgrad stoßen Mapper
an die Grenzen dessen, was in OSM-Editoren sinnvoll zu bearbeiten ist. Wir haben daher eine offene
Plattform zum Austausch frei lizenzierter Modelle geschaffen.

Dinge der realen Welt~-- von der Parkbank bis zur Burg~-- können in einem dafür ausgelegten
3D-Editor erstellt werden und über das \emph{3D Model Repository} von jedermann mit OpenStreetMap
verknüpft werden.%
}

\newtimeslot{12:15}
% time: 2018-03-22 12:15:00
\abstractAPH{Marco Lechner}%
{Fortgeschrittene OpenLayers-Overlays im BfS-Web-Client\vspace{0.2em}}%
{Von der Visualisierung bis zum Druck}%
{%
Um den radiologischen Notfallschutz weiterzuentwickeln, setzt das Bundesamt für Strahlenschutz (BfS)
auf eine Open-Source-Strategie. Im Web-GIS des neuen IMIS\,3 werden OpenLayers, GeoExt und
MapfishPrint eingesetzt und zur Weiterentwicklung der Projekte beigetragen. Der Vortrag präsentiert
den fortgeschrittenen Einsatz von OpenLayers"=Overlays im Webclient von interaktiven
Kartodiagrammen, in denen Zeitreihen, Tabellen und Balkendiagramme dargestellt werden, bis zum Druck
durch MapfishPrint~3.

Unter github.com/OpenBfS veröffentlicht das BfS den Quellcode.
}

% time: 2018-03-22 12:15:00
\abstractZwei{Pirmin Kalberer}%
{Styling und Publikation von \mbox{Vektortiles}}%
{}%
{%
Vektortiles haben das Potential, die bewährten Rasterkarten in vielen Bereichen abzulösen oder
mindestens maßgeblich zu ergänzen. Für das Styling hat sich Mapbox~GL~JS als Industriestandard
etabliert. Neben dem Viewer und den nativen SDK für Android, iOS, macOS, Node.js und Qt von Mapbox
unterstützt auch OpenLayers den Import von Mapbox-GL-Styles.  Der Vortrag bietet eine Einführung
in das Styling-Format von Mapbox~GL~JS und gibt Tipps zur Publikation von Vektortiles.%
}

% time: 2018-03-22 12:15:00
\abstractVier{Tobias Knerr}%
{3D~-- mehr als Gebäude}%
{OSM2World jenseits von Simple 3D Buildings}%
{%
Die Fähigkeit zur Darstellung von 3D-Gebäuden ist heute beinahe schon Standard. Für eine umfassende
dreidimensionale Abbildung der Welt müssen aber auch viele andere Objekte berücksichtigt werden und
OpenStreetMap bietet dafür beste Voraussetzungen. Am Beispiel des freien 3D-Renderers OSM2World
werden die Möglichkeiten der OSM-Daten für das 3D-Rendering jenseits von Gebäuden gezeigt.%
}


\newtimeslot{13:40}
% time: 2018-03-22 13:40:00
\abstractAPH{Felix Kunde}%
{PostGIS v2+}%
{Überblick an Funktionen der letzten Releases}%
{%
Mit jeder neuen Version unserer Lieblingsgeodatenbank PostGIS kommen neue spannende Funktionen
hinzu. Auch das darunterliegende PostgreSQL entwickelt sich beständig weiter. Oft merkt man sich
ein, zwei Highlights pro Release und übersieht bzw. vergisst den Rest. Dieser Vortrag lässt die
neuen Features der einzelnen PostGIS"=Releases seit der Version 2.0 im Jahr 2012 Revue passieren.%
}

% time: 2018-03-22 13:40:00
\abstractZwei{Numa Gremling}%
{Webmapping und Geoverarbeitung~-- Turf.js}%
{}%
{%
Turf.js ist eine Open-Source-JavaScript-Bibliothek, die mit oft nur sehr wenigen Befehlen
ermöglicht, klassische Geoverarbeitungswerkzeuge im Browser auszuführen. Das Format GeoJSON
ermöglicht das clientseitige Verarbeiten und Analysieren von Geodaten und spart Ihnen die
Einrichtung einer komplexen serverseitigen Infrastruktur. Komfortabler geht es kaum: Turf einbinden,
wenigen Zeilen Code schreiben und in Sekundenschnelle komplexe ortsbezogene Fragen beantworten. Und
das alles lokal in Ihrem Browser und sogar offline!%
}

% time: 2018-03-22 13:40:00
\abstractVier{Raffael}%
{Open Data im ÖPNV}%
{}%
{%
In der Präsentation wird zu Beginn ein Überblick über den Stand von Open Data im ÖPNV gegeben --
Schwerpunkt dabei sind Fahrplandaten in Deutschland. Im zweiten Teil der Präsentation werden
Anwendungen vorgestellt die mit offenen ÖPNV-Daten arbeiten.%
}

\newtimeslot{14:15}
% time: 2018-03-22 14:15:00
\abstractAPH{Pirmin Kalberer}%
{GeoPackage als Arbeits- und \mbox{Austauschformat}}%
{}%
{%
In GeoPackage-Dateien können sowohl Vektor- als auch Rasterdaten samt der zugehörigen Metainformation
gespeichert werden. Damit können Geodaten einfach ausgetauscht und auch auf mobilen Geräten
effizient genutzt werden.  Der Vortrag zeigt die Einsatzmöglichkeiten von GeoPackage mit dem Fokus
auf QGIS und gibt einen aktuellen Überblick über GeoPackage-Extensions.%
}

% time: 2018-03-22 14:15:00
\abstractZwei{Christian Mayer}%
{Wegue -- Web-GIS-Anwendungen mit OpenLayers und Vue.js}%
{}%
{%
Wegue ist eine Open-Source-Software zum Erstellen von modernen leichtgewichtigen
Web"=GIS"=Client"=Anwendungen. Die Basis dafür sind die beiden JavaScript"=Frameworks OpenLayers und
Vue.js.

Wegue verknüpft diese beiden Bibilotheken zu einer konfigurierbaren Vorlage für WebGIS-Anwendungen
aller Art und stellt wiederverwendbare UI-Komponenten (z.\,B. Layer-Liste, FeatureInfo-Dialog, etc.)
bereit. Somit können Anwender und Entwickler schnell zu einem ansprechendem und modernen
Web-GIS-Client zur Ver\-öf\-fent\-li\-chung und Nutzung von Geodaten gelangen.%
}

% time: 2018-03-22 14:15:00
\abstractVier{Christoph Hormann}%
{Darstellungsorientierte Generalisierung von offenen Geodaten}%
{}%
{%
Dieser Vortrag stellt die jüngsten Entwicklungen im Bereich der darstellungsorientierten
automatischen Generalisierung von offenen Geodaten vor. Ziel hiervon ist es, die Qualität
automatisiert regelbasiert produzierter Kartendarstellungen in digitalen Karten zu verbessern.
Anhand von Beispielen werden die Neuerungen und zusätzliche Anwendungsfelder vorgestellt und sowohl
Chancen als auch die Herausforderungen des darstellungsorientierten Ansatzes wie auch der
Verwendung offener Geodaten erläutert.%
}

\newtimeslot{14:50}
% time: 2018-03-22 14:50:00
\abstractAPH{Felix Kunde}%
{Kompakte Datenbankschemata für \mbox{dynamisch} erweiterbare GML Application Schemas\vspace{0.3em}}%
{Die neue Version der 3DCityDB zeigt, wie es gehen kann}%
{%
Durch größere Verfügbarkeit von 3D-Geodaten wächst die Akzeptanz für CityGML und der Bedarf nach
Domänen-spezifischen Erweiterungen des Standards (ADEs), z.\,B. Lärmkartierung oder
Energiemanagement. Der Vortrag gibt einen Ausblick auf die neue Version der \emph{3D City Database}, die
beliebige ADEs dynamisch einbinden kann, ohne dass das PostGIS-Datenbankschema zu komplex und
schwerfällig wird.%
}

% time: 2018-03-22 14:50:00
\abstractZwei{Christian Mayer, Marc Jansen}%
{Adult.js -- JavaScript ist erwachsen geworden!}%
{}%
{%
Die Zeiten, in denen JavaScript als eine reine Skriptsprache zur dynamischen Anpassung von HTML-Elementen
in Browsern genutzt wurde, sind lange vorüber. Vielmehr werden mittlerweile komplexe Applikationen in
Java\-Script programmiert, sowohl im Client als auch auf dem Server.

Der Vortrag gibt eine Übersicht über die heutigen Möglichkeiten der Geodatenverarbeitung im Client
und Server mittels JavaScript. Außerdem wird die aktuelle Professionalisierung in der
JavaScript-Entwicklung beleuchtet und bewertet.%
}

% time: 2018-03-22 14:50:00
\abstractVier{Thomas Skowron}%
{Pipelinebasierte Erzeugung von Karten}%
{Geodaten verarbeiten ohne Datenbanksystem}%
{%
Im OpenStreetMap-Umfeld werden Daten meist erst in eine Datenbank geladen, um diese hiernach wieder zu
extrahieren. Im Zuge dessen entstehen bei großen Datensätzen hierbei häufig Flaschenhälse, die eine
effiziente Verarbeitung verhindern. Dieser Vortrag schlägt Methoden vor, um Daten sequentiell in
einer Pipelinestruktur zu verarbeiten, um ressourcenschonend und schneller als bestehende Lösungen
Daten zu verarbeiten, filtern und zu transformieren.%
}

\newtimeslot{15:45}
\abstractAPH{}{Lightning Talks}{}{}

\abstractZwei{Frederik Ramm}%
{Lügen mit Statistik, OpenStreetMap-Edition}%
{Missverständnisse und Fehlinterpretationen mit OSM-Metadaten}%
{%
In diesem Vortrag geht es nicht um die Geodaten in OpenStreetMap, sondern um die Daten hinter den
Daten. Wer hat was wann eingetragen, wie viele Mapper arbeiten eigentlich an den Daten, und welche
Daten sammeln die Mapper am liebsten? Immer wieder kommen Außenseiter hier zu drastischen
Fehleinschätzungen. Dieser Vortrag zeigt ein paar richtige und falsche Statistiken und erklärt, wie
man es richtig macht.%
}

% time: 2018-03-22 15:45:00
\abstractVier{Arndt Brenschede}%
{Energieeffizientes PKW-Routing mit OpenStreetMap}%
{}%
{%
Energieeffizientes PKW-Routing, manchmal auch Eco-Routing genannt, ist von der Idee nicht neu, aber
kaum verbreitet und begrifflich undefiniert. Dieser Beitrag schafft hier Klarheit, zeigt das
Potential für die Elektromobilität, diskutiert die besonderen Anforderungen, die energieeffizientes
Routing an die Qualität von Straßenkarten stellt und untersucht die Eignung von OpenStreetMap für
diesen Anwendungsbereich.%
}

\newtimeslot{16:20}
\abstractAPH{Johannes Kröger}%
{Karten aus QGIS ins Buch, Web oder auf die Leinwand}%
{Eine Übersicht der vielseitigen Exportmöglichkeiten von QGIS}%
{%
Neben den mitgelieferten Funktionen bietet QGIS dank seines umfangreichen Pluginkatalogs eine
Vielzahl von Möglichkeiten Kartenprojekte in unterschiedlicher Art und Weise und für
unterschiedlichste Zwecke zu exportieren. Etwa per automatisierter "`Stapelverarbeitung"', als
interaktive Webkarten, Videos oder auch 3D-Viewer. Die Atlas-Erzeugung und die Plugins HTML Image
Map Creator, qgis2web, QTiles, Time Manager und qgis2threejs stellen diese Optionen zur Verfügung.%
}

\abstractZwei{Robin Luckey}%
{Master Portal}%
{Das Open-Source Web-GIS der Stadt Hamburg}%
{%
  Das Masterportal ist eine OGC-konforme, Open-Source-Web-GIS-Lösung (MIT-Lizenz) zur Generierung von digitalen
Kartenanwendungen. Es basiert auf BackboneJS und OpenLayers und wird aktiv von der Stadt Hamburg
weiterentwickelt. Es ermöglicht ohne Programmierkenntnisse und unter geringem Aufwand thematische
Kartenanwendungen zu erstellen, außerdem ist es leicht Erweiterbar und kann es als Framework zur
Erstellung von komplexen Kartenanwendungen genutzt werden.%
}

% time: 2018-03-22 16:20:00
\abstractVier{Michael Reichert}%
{Eisenbahnrouting mit GraphHopper}%
{}%
{%
In diesem Vortrag werden Anpassungen an GraphHopper vorgestellt, mit denen ein Routing auf
Eisenbahngleisen möglich ist. Der Vortrag geht darauf ein, welche Anpassungen vorgenommen werden
müssen und ist daher in Teilen auch als Anleitung zum Schreiben von FlagEncodern zu verstehen.

Ganz einfach ist das Routing auf Eisenbahngleisen jedoch nicht. Zwar wird jedes Gleis als ein Way in
OSM erfasst, welches mit den anderen Gleisen verbunden ist. Manche Eigenschaften von
Schienenfahrzeugen lassen sich nicht so einfach abbilden. Der Vortrag wird in seinem Ausblick daher
kurz darlegen, was für ein besseres Routing noch fehlt.%
}


\newtimeslot{16:55}
% time: 2018-03-22 16:55:00
\abstractAPH{Marco Hugentobler}%
{Datenqualität sicherstellen mit QGIS}%
{}%
{%
QGIS bietet eine Reihe von Funktionen, um die Geometrien eines Datensatzes zu überprüfen und zu
korrigieren. Der Vortrag gibt einen Überblick über die beiden Plugins \emph{Geometrychecker} und
\emph{Topologychecker} und zeigt Gemeinsamkeiten und Unterschiede auf.%
}

% time: 2018-03-22 16:55:00
\abstractZwei{Martin Dresen}%
{BKG WebMap -- ein OpenLayers\,4-Framework zur einfachen Erstellung interaktiver Webkarten}%
{}%
{%
Die BKG WebMap des Bundesamts für Kartographie und Geodäsie ist eine JavaScript-Bibliothek, die
verschiedene Funktionen zur einfachen Erstellung interaktiver Karten bereithält. Sie wurde jetzt auf
der Basis von OpenLayers~4 neu entwickelt und wird auf der FOSSGIS-Konferenz erstmalig präsentiert.%
}

% time: 2018-03-22 16:55:00
\abstractVier{Hartmut Holzgraefe}%
{OSM Daten mit Mapnik und Python rendern}%
{Eine kurze Einführung }%
{%
Mapnik ist eine Open-Source-Bibliothek zur Erstellung von Karten, wie z.\,B. auf openstreetmap.org
zu sehen. Mapnik bietet eine eigene XML-basierte Stylesheet-Sprache und verarbeitet Daten aus
verschiedenen Geodaten-Quellen.%
}

\newtimeslot{17:30}
% time: 2018-03-22 17:30:00
\abstractAPH{Otto Dassau}%
{Geometrie- und Topologiefehler finden und korrigieren}%
{Möglichkeiten mit QGIS und GRASS GIS}%
{%
Ob man nun Daten selber generiert oder Daten von Anbietern verwendet -- man kommt nicht umhin, diese
auf geometrische und topologische Fehler zu prüfen und diese zu bereinigen. 

In diesem Vortrag werden wir etwas hinter die Kulissen schauen. Welche Unterstützung bieten die
Methoden der GEOS-Bibliothek im Vergleich zu QGIS' eigenen Algorithmen? Wir werden QGIS-Plugins zur
Geometrieprüfung vorstellen und deren Ergebnisse vergleichen und Hintergründe beleuchten.
Alternativen werden aufgezeigt und andere Lösungswege skizziert, wie z.\,B. über das GRASS-Plugin.%
}

% time: 2018-03-22 17:30:00
\abstractZwei{Armin Retterath}%
{INSPIRE Downloaddienste}%
{Praktische Erfahrungen der letzten vier Jahre}%
{%
Im Vortrag werden die neuesten Entwicklungen bezüglich der Umsetzung und Nutzung von
INSPIRE-Downloaddiensten in den Ländern Hessen, Rheinland-Pfalz und Saarland anhand praktischer
Beispiele vorgestellt. Dabei wird insbesondere der immense Mehrwert für die Praxis ersichtlich, den
die Standardisierung durch die INSPIRE-Richtlinie gebracht hat.%
}

% time: 2018-03-22 17:30:00
\abstractVier{Petr Pridal}%
{OpenMapTiles}%
{Revolution in selbstgehosteten Karten}%
{%
Ihre eigenen weltweiten Straßenkarten auf einem lokalen Computer oder auf privaten und öffentlichen
Clouds hosten? Ja, dank Vektorkacheln, Open-Source-Software und Open Data. Erfahren Sie, wie Sie
Karten mit eigenem Design in Ihren Websiten und mobilen Apps, oder in QGIS und ArcGIS anzeigen
können. Generieren Sie eigene Vektorkacheln und hosten Sie diese selbst. Eigenen Geodaten können
integriert werden. Das OpenMapTiles-Projekt wird bereits von Siemens, IBM, Bosch, Amazon, SBB und
anderen angewendet.%
}

\newsmalltimeslot{18:00}
\abstractZwei{}%
{FOSSGIS-Mitgliederversammlung}%
{}%
{%
Alle Mitglieder sind eingeladen, teilzunehmen und sich zu beteiligen.%
}
