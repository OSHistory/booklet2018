\renewcommand{\conferenceDay}{\freitag}
\textcolor{red}{Werbung}
\newpage
\newtimeslot{09:00}
% time: 2018-03-23 09:00:00
\abstractAPH{Axel Schaefer}%
{Neues in Metador, kurz vor der finalen Version}%
{Konfomität, Anpassungen, CSW}%
{%
Was lange währt wird auch released. 2018 wird die Metadatensoftware Metador
in der neuen Version veröffentlicht werden. Monatelange Arbeit flossen in
die Software rein, um viele Features zu implementieren: von einem CSW-Plugin
bis hin zu einer einfachen Erstellung von Formularen, der Validierung von
Formularen und einigem mehr.

Der Vortrag zeigt praktisch und nachvollziehbar, wie Metador bedient wird,
der CSW funktioniert und wie eigene Formulare gebaut werden.%
}

% time: 2018-03-23 09:00:00
\abstractZwei{Tobias Werner}%
{Einsatz von SpatiaLite auf teilautonomen Unterwasserfahrzeugen}%
{}%
{%
Unterwasserfahrzeuge unterstützen zunehmend Inspektions- und Forschungsarbeiten
in Gewässern. Dabei verfügen sie nur über eingeschränkte Möglichkeiten zur
Datenübertragung. Dieser Vortrag befasst sich mit der Speicherung, Verwaltung
und Organisation von Beobachtungsdaten nach raumzeitlichen Kriterien auf
teilautonomen Unterwasservehikeln. Kernelemente stellen SpatiaLite und OGCs
SensorThings-API dar.%
}

% time: 2018-03-23 09:00:00
\abstractVier{Daniel Karla}%
{Geschwindigkeitsoptimierter Ansatz zur Analyse der ärztlichen Versorgungsqualität}%
{}%
{%
Im Rahmen dieses Beitrags werden optimierenden Aspekte betrachtet, welche bei
der Umsetzung eines Prototyps für die Analyse der ärztlichen
Versorgungsqualität notwendig wurden. Der hier gewählte Ansatz zur Untersuchung
der Versorgungsqualität basiert auf der Nutzung von Erreichbarkeitspolygonen,
welche mit Planungsgebieten der Kassenärztlichen Vereinigung und Wohngebieten
verschnitten werden.%
}

\newtimeslot{09:35}
% time: 2018-03-23 09:35:00
\abstractAPH{Felix Kunde}%
{Von Mobilitätsdiensten, Datenplattformen und Handwerkern}%
{}%
{%
Seit einigen Jahren fördert der Bund den Aufbau von offenen Plattformen für
Mobilitätsdaten und Diensten, um den zukünftigen Markt nicht komplett großen
Tech-Unternehmen wie Google, HERE oder Uber zu überlassen. Der Vortrag
berichtet von Erfahrungen aus einem dieser Projekte und geht neben den
entwickelten Ansätzen auch auf die Schwierigkeiten beim Integrieren von
Verkehrstelemetrie-Daten sowie dem Ansprechen von möglichen Zielgruppen ein.%
}

% time: 2018-03-23 09:35:00
\abstractZwei{Stefan Blumentrath}%
{Der Nordische Süßwasser Atlas}%
{Effizients Management von Forschngsdaten mit FOSS}%
{%
Der Nordische Süßwasser Atlas (NOFA) ist eine räumliche
Forschungsdateninfrastruktur auf der Basis freier und Open-Source-Software.
Er soll den Forschern effiziente Arbeitsflüsse von der Datenerfassung, über
die Analyse bis hin zur Präsentation gewährleisten und die serverseitige
Analyse größer Datenmengen ermöglichen. Ziel ist es, Risiko und Effekte der
Einführung, Etablierung und Weiterverbreitung invasiver Arten abschätzen zu
können.%
}

% time: 2018-03-23 09:35:00
\abstractVier{Heinrich Lorei}%
{Spielbasierte Ansätze in der Geodatenerfassung}%
{Zum Einfluss von Spielelementen auf Anreizmechanismen von OpenStreetMap-Nutzern}%
{%
Die Geodatenerfassung mittels Elementen der Gamification bietet vielfältige
Mehrwerte. Zugehörige Konzepte können genutzt werden, um passive Nutzer von
OpenStreetMap (OSM) als aktive Datenproduzenten zu gewinnen. Bereits
vorliegende Entwurfsmuster aus dem Game Design können im Rahmen der
Geodatenerfassung in OSM angewendet werden, um diese abwechslungsreicher zu
gestalten. Im Vortrag werden erste Ergebnisse einer in Heidelberg
durchgeführten Fallstudie vorgestellt, in der verschiedene Varianten eines
mobilen Prototypen miteinander verglichen werden.%
}

\newtimeslot{10:10}
% time: 2018-03-23 10:10:00
\abstractAPH{Marco Pochert}%
{GeoNetwork als Konfigurationsoberfläche eines dynamischen Geoportals}%
{}%
{%
Um den radiologischen Notfallschutz weiterzuentwickeln, setzt das Bundesamt für
Strahlenschutz (BfS) auf eine Open-Source-Strategie. Ein Teilprojekt ist
hierbei das BfS-Geoportal, das seit Anfang des Jahres 2017 online ist.

Der Vortrag zeigt wie GeoNetwork, neben der Verwaltung von Metadaten, mithilfe
von Konfigurationsparametern auf thematischer Ebene für individuelle
Visualisierungen komplexer Messdaten genutzt werden kann. Aus
fachadministrativer Sicht soll die Möglichkeit bestehen, auf Änderungen der
Datengrundlage ad hoc reagieren zu können.%
}

% time: 2018-03-23 10:10:00
\abstractZwei{Niny Zamora}%
{The landslide map of Bogota updating}%
{}%
{%
  \emph{Dieser Vortrag wird in englischer Sprache gehalten.}

  \begin{otherlanguage}{english}
    Erdrutsche stellen eine große Bedrohung für Bogota dar. Das starke städtische
    Wachstum, die steile Landschaft und die heftigen Regenfälle spielen eine
    wichtige Rolle. Die analysierte Fläche umfasste über 16\,000 Hektar. Dies ergab
    ein sehr großes Datenvolumen, deshalb brauchte man effiziente und robuste
    Softwaretools. Des\-wegen war GRASS GIS das beste Tool, um die diversen
    Geoprozesse durchzuführen. Die Ergebnisse dieses Modells werden in den nächsten
    15 Jahren von der Stadtverwaltung Bogota genutzt um die Sicherheit ihrer Bürger
    zu gewährleisten.%
  \end{otherlanguage}
}

% time: 2018-03-23 10:10:00
\abstractVier{Christian Bittner}%
{OpenStreetMap in Israel und Palästina -- zwei ungleiche Geschichten}%
{}%
{%
Dieser Vortrag präsentiert Forschungen zu OSM in Israel und Palästina. Es zeigt
sich, dass OSM in Israel, ähnlich wie in vielen europäischen Ländern, von einer
sehr aktiven lokalen Community getragen wird. Eine palästinensische
OSM-Community hat sich jedoch bislang nicht gebildet. In Israel sind die
OSM-Daten daher tendenziell auch dichter und reichhaltiger als in Palästina.
Neben sozioökonomischen Strukturen scheint hier auch die Agenda von OSM eine
Rolle für diese Ungleichheiten zu spielen.%
}


\newtimeslot{11:05}
% time: 2018-03-23 11:05:00
\abstractAPH{Jörg Thomsen}%
{GISInfoService}%
{Ein verteiltes Web-GIS}%
{%
Am Beispiel des Portals GISInfoService.de soll ein komplexes Web-GIS
vorgestellt werden, wobei sich die Komplexität gar nicht so sehr auf die
Funktionen bezieht, die bereit gestellt werden, sondern auf die
Gesamtarchitektur mit einer Verteilung über mehrere Server und einen hohen
Automatisierungsgrad bei der Wartung. Das System besteht besteht aus mehreren
Installationen. Da das Web-GIS über viele zentral gepflegte Kartenebenen
verfügt, wurde die Verteilung der Konfiguration auf die verschiedenen
Installationen automatisiert.%
}

% time: 2018-03-23 11:05:00
\abstractZwei{Athina Trakas}%
{Tech trends und Neues aus dem OGC}%
{}%
{%
Der Vortrag gibt einen Einblick in aktuell diskutierte Themen im OGC. Neue und
auch disruptive Technologien spielen eine immer größere Rolle, auch im
Geo-Bereich. Wie adressiert die Standardisierungscommunity diese Trends, was
sind die "`Hot Topics"', welche Rolle spielen bestehende und neu zu
definierende offene Standards, um Daten zugänglich und nutzbar zu machen? Und
welche Rolle spielt die Freie-Software-Community, die die OGC-Standards
implementiert?%
}

% time: 2018-03-23 11:05:00
\abstractVier{Pascal Neis}%
{Eine konfigurierbare Karte mit Verbotszonen für Drohnenflieger auf Basis von OpenStreetMap-Daten}%
{Pascal Neis, Hans-Jörg Stark}%
{%
Der Vortrag gibt einen Überblick über die Gesetzeslage in Deutschland und der
Schweiz. Dabei wird speziell auf Richtlinien und Flugverbotszonen, die von
Piloten beachtet werden müssen, eingegangen. Auf Basis der Daten des
OpenStreetMap-Projekts wird anschließend ein Programm vorgestellt, welches auf
die Eigenschaften der eigenen Drohne angepasst werden kann. Das Ergebnis des
Programmes ist eine interaktive Karte für den Webbrowser, die alle
konfigurierten und zu beachtenden Verbotszonen für ein beliebiges Gebiet
anzeigen.%
}

\newtimeslot{11:40}
% time: 2018-03-23 11:40:00
\abstractAPH{Till Adams}%
{QGIS, GeoServer und SHOGun im Zusammenspiel}%
{}%
{%
Der Vortrag zeigt beispielhaft, wie ein umfassendes Gesamtkonzept einer solchen
Geodatenarchitektur rein auf Basis von freier Software aussehen kann. 

Dazu wird als Beispiel ein Zusammenspiel der Komponenten QGIS, GeoServer und
SHOGun vorgestellt. QGIS fungiert dabei als Software am GIS-Arbeitsplatz und
wird auch zur Kartenerstellung benutzt. Diese werden über ein Plugin in
GeoServer und von dort letztendlich über das Web-GIS-Framework SHOGun im Web an
bestimmte Nutzergruppen veröffentlicht.%
}

% time: 2018-03-23 11:40:00
\abstractZwei{Dirk Stenger}%
{TEAM Engine -- eine Validierungs-Engine für OGC-Geodienste und -formate}%
{Wie kann ich von diesem Tool profitieren?}%
{%
Die TEAM Engine ist eine Engine, mit der Entwickler und Anwender Geodienste,
wie WFS und WMS, und Geoformate, wie GML oder GeoPackage, testen können.  Es
werden die aktuellen Entwicklungen der TEAM Engine und der dazugehörigen
Testsuites vorgestellt. Zudem wird ein Ausblick gegeben, was die Schwerpunkte
der Weiterentwicklung in der Zukunft sein werden.

Dieser Vortrag beantwortet abschließend die Frage, wie einzelne Nutzer mit
verschiedenen Interessenschwerpunkten von der TEAM Engine und den aktuellen und
zukünftigen Entwicklungen profitieren können.%
}

% time: 2018-03-23 11:40:00
\abstractVier{Roland Olbricht}%
{Freies Undo in OSM}%
{Overpass-API 0.7.55}%
{%
Die Funktionalität, die Achavi benötigt, bietet erst die neue Version 0.7.55
der Overpass-API. Weiteres Ziel ist, ein Undo von unerwünschten Änderungen
uneingeschränkt auch dann zu ermöglichen, wenn es darauf basierend schon
Änderungen gegeben hat. Dafür ist ein eigenes Co-Datenmodell erforderlich.

Für den Hauptteil des Vortrags sind alle Teilnehmer dann eingeladen, ihr
Lieblings-OSM-Problem mitzubringen, um es live mit der Overpass-API zu lösen.%
}

\newtimeslot{12:15}
% time: 2018-03-23 12:15:00
\abstractAPH{Johannes Weskamm}%
{Karten gestalten im GeoServer~-- SLD, CSS und MBStyles}%
{}%
{%
Der GeoServer bietet nicht nur die Möglichkeit, Geodienste zu verschiedensten
Fachthemen OGC-konform zu publizieren, sondern auch diese entsprechend
auszugestalten.

In älteren Versionen von GeoServer wurden die Styles ausschließlich im
XML-basierten SLD geschrieben. Mittlerweile werden auch CSS, YSLD und
Mapbox-Styles als Alternative angeboten.  Auch OpenLayers~3 unterstützt
inzwischen Mapbox-Styles, die mit verschiedenen Editoren bearbeitet werden
können. Der Vortrag wird die Unterschiede und Möglichkeiten der Formate anhand
von Beispielen aufzeigen.%
}

% time: 2018-03-23 12:15:00
\abstractZwei{Torsten Friebe}%
{Aktuelles aus dem Deegree-Projekt}%
{Neues in 3.4 und Weiterentwicklungen für INSPIRE}%
{%
Der Vortrag zeigt die verbesserte Unterstützung für INSPIRE in der aktuellen
deegree-Version anhand von Beispielen auf. Hierbei wird auch auf die von
Deegree unterstützten unterschiedlichen Datenformate und -quellen eingegangen.
Spezielle Konfigurationen für INSPIRE werden an praktischen Beispielen
dargestellt. Neben dem aktuellen Stand des Communityprojekts wird auch
aufgezeigt, welche zukünftigen Entwicklungen derzeit geplant sind und wie der
Stand von häufig nachgefragten Features ist.%
}


%%%%%%%%%%%%%%%%%%%%%%%%%%%%

% time: 2018-03-23 12:15:00
\abstractVier{Christopher Lorenz}%
{Gut gemeint -- schlecht umgesetzt}%
{Häufige Schönheitsfehler in der OpenStreetMap, wo sie herkommen und wie man sie vermeiden kann.}%
{%
OpenStreetMap lebt von der Community, die Community entsteht aus Menschen und
Menschen machen Fehler. Der Vortrag zeigt eine Auswahl an Fehlern bzw.
Schönheitsfehlern auf, die auch auf komplexe Strukturen in der realen und
OSM-Welt zurückzuführen sind. Es werden auch Tipps gegeben, wie man Fehler
vermeiden oder finden kann bzw. wie man sich beim Auffinden von Fehlern
verhalten sollte.%
}

\newtimeslot{13:40}
% time: 2018-03-23 13:40:00
\abstractAPH{Stefan Kuethe}%
{OpenLayers 4 R}%
{Seamlessly bridge R and OpenLayers.js}%
{%
Dieser Talk stellt das R-Paket \emph{OpenLayers 4 R} vor, das eine nahtlose
Einbindung von OpenLayers in R ermöglicht. Die Visualisierungen können dabei
direkt in der R-Konsole, in R-Markdown-Dokumenten oder
Shiny-Web-Applikationen angezeigt werden. Die Einbindung erfolgt in der
gewohnten R-Syntax mit wenigen Zeilen Code.%
}

% time: 2018-03-23 13:40:00
\abstractZwei{Benjamin Pross}%
{WPS 2.0 REST/JSON-Extension}%
{}%
{%
Wir geben einen Ausblick auf die REST/JSON-Erweiterung zum Standard Web
Processing Service~2.0. Die Standardisierung wird im OGC durch die
WPS-2.0-Standard-Arbeitsgruppe vorangetrieben. Parallel wird eine
Beispielimplementierung von 52°\,North entwickelt. Neben dem formalen
Standarddokument ist die Erweiterung als API beschrieben, die der
OpenAPI-3.0-Spezifikation folgt. Wir stellen die API und die Implementierung,
sowie die geplante Roadmap bis zur Veröffentlichung vor.%
}

% time: 2018-03-23 13:40:00
\abstractVier{Pascal Neis}%
{"`Ich weiß was du letzten Sommer gemappt hast!"' -- Datenspuren im OpenStreetMap-Projekt}%
{}%
{%
Dieser Vortrag soll einen Überblick geben, wo und welche Informationen von
Beitragenden im OSM-Projekt und dessen Unterprojekten gespeichert werden. An
verschiedenen Beispielen wird exemplarisch gezeigt, wie sich diese
Informationen über Beitragende verwenden und verknüpfen lassen. Abschließend
werden Vor- und Nachteile der gespeicherten Datenspuren diskutiert. Dabei
werden ebenfalls einfache Empfehlungen präsentiert, die Mitglieder beachten
können, wenn sie Bedenken bzgl. ihrer gespeicherten Informationen und
Datenspuren haben.%
}

\newtimeslot{14:15}
% time: 2018-03-23 14:15:00
\abstractAPH{Daniel Koch}%
{React meets OpenLayers}%
{Vorstellung von und Anwendungsbeispiel mit react-geo}%
{%
React ist derzeit eines der meistgenutzten Frameworks zur Entwicklung von
Web-UI-Komponenten. OpenLayers ist eine weit verbreitete Bibliothek, um
webbasierte Kartenanwendungen zu erstellen. In diesem Vortrag werden wir die
junge Open-Source-Bibliothek \emph{react-geo} vorstellen, welche die Vorteile beider 
Bibliotheken verbindet.

Nach einer kurzen Vorstellung von React und OpenLayers werden wir die Merkmale
und Komponenten von react-geo demonstrieren. Eine Präsentation einer react-geo
Applikation sowie ein Vergleich mit Alternativen runden den Vortrag ab.%
}

% time: 2018-03-23 14:15:00
\abstractZwei{Juiwen Chang}%
{Aufgabenorientierte Datenklassifikation in Choroplethen-Karten für die Erhaltung lokaler Extremwerte}%
{}%
{%
In thematischen Karten werden die Werte einzelner Regionen zur besseren
Übersicht oft in verschiedene Klassen eingeteilt und entsprechend farblich
kodiert, um einen besseren Überblick zu erhalten. Der visuelle Eindruck solcher
Choroplethen-Karten wird dabei durch die gegebene Werteverteilung, die Methode
zur Datenklassifikation und die Farbwahl bestimmt. Konventionelle
Klassifizierungsmethoden funktionieren datenbasiert und beachten dabei nicht
den räumlichen Kontext der Daten. Verfolgen wir eine aufgabenorientierte
Klassifikation.%
}


%%%%%%%%%%%%%%%%%%%%%%%%%%%%

% time: 2018-03-23 14:15:00
\abstractVier{Falk Zscheile}%
{Datenschutz und die Daten der zu OpenStreetMap Beitragenden}%
{Datenhaltung im Lichte der Datenschutzgrundverordnung}%
{%
Der Beitrag analysiert die Nutzerdatenhaltung von OpenStreetMap mit Blick auf
die Vorgaben der Datenschutzgrundverordnung (DSGVO) und den sich daraus für das
OpenStreetMap-Projekt ergebenden Anpassungsbedarf.%
}

\newtimeslot{14:50}
% time: 2018-03-23 14:50:00
\abstractAPH{Rouven Volkmann}%
{Bereitstellung eines Webservices von globalen, kontinuierlich einfließenden Satellitendaten hoher Auflösung am Beispiel von Sentinel-2}%
{Ein Erfahrungsbericht}%
{%
Täglich werden mehr als 7000 Sentinel-2-L1C-Produkte mit jeweils rund 500
Megabyte aufgenommen. Anders ausgedrückt: 1,2~Petabytes Daten pro Jahr. Ich
habe die Aufgabe einen Full-Resolution-Webservice dieser Daten im Netz zur
Verfügung zu stellen und dauerhaft aktuell zu halten. Dabei sind die
Anforderungen an den Service eine hohe Datenqualität, gute Performance und ein
geringer Speicherbedarf. Dieser Vortrag ist ein Erfahrungsbericht der
ausschließlich mit Open-Source-Tools bewerkstelligten Umsetzung.%
}


%%%%%%%%%%%%%%%%%%%%%%%%%%%%

% time: 2018-03-23 14:50:00
\abstractZwei{Otto Dassau}%
{Historisierung von Vektorobjekten mit QGIS und PostGIS}%
{}%
{%
Das Thema Historisierung von Vektorobjekten ist nicht neu, gewinnt aber seit
einiger Zeit immer mehr an Bedeutung, vor allem im Hinblick auf eine
gerichtsfeste Dokumentation. Dafür ist es notwendig, in der Vergangenheit
liegende Datenstände mit einer einwandfreien Historisierung der Daten
reproduzieren zu können.

Das Geoinformatikbüro Dassau hat im Rahmen von zwei aktuellen Projekten einen
leichtgewichtigen, skalierbaren Ansatz zur Historisierung und Versionierung von
beliebig großen Vektordaten auf Basis von QGIS und PostGIS programmiert.%
}

\abstractVier{Peter Barth}%
{OSM-Quiz}%
{Wie gut kennst du OSM?}%
{Das OSM-Quiz bietet als Fortsetzung des Events der letzten Jahre wieder spannende Fragen zu
interessanten Fakten. Jeder ist herzlich eingeladen mitzuraten um sein Wissen im Umfeld von
OpenStreetMap zu testen.}

\newsmalltimeslot{15:30}
\abstractAPH{Vorstand}{Abschlussveranstaltung}{}{}

\vspace{-1.0\baselineskip}
\abstractAPH{}{Sektempfang}


\newpage
\section*{OSM-Samstag}
\pagestyle{cropmarksstyle}
\label{osm-samstag}
Am Samstag, den 24. März findet von 09:30 bis 18:00 Uhr im Roten Saal und den angrenzenden Räumen der OSM-Samstag statt.
Der OSM-Samstag ist als eine Unkonferenz (Barcamp) und Mappertreffen gedacht.
Es richtet sich an Mapper, Entwickler und OSM-Interessierte im näheren und
weiteren Umkreis von Bonn und natürlich an alle Teilnehmer der am Vortag
endenden FOSSGIS-Konferenz. Um eine Anmeldung am Welcome Desk wird gebeten.
Folgende Themen wurden im OSM-Wiki schon vorgeschlagen:
\begin{itemize}
  \RaggedRight
  \setlength{\itemsep}{-1pt}
  \item \emph{Christopher:} Qualitätssicherung und Erfassung von Adressen in OSM
  \item \emph{Christopher:} Diskussion meines Vortrages \emph{Gut gemeint -- Schlecht umgesetzt}, ggf. Präsentation Langversion 
  \item \emph{Kevin:} ÖPNV: Nutzung von DINO-Daten zum Vergleich mit OSM
  \item \emph{Kevin:} Tagging usw. von Fahrradknotenpunktnetzwerken wie z.B. radrevier.ruhr (Diskussion)
  \item \emph{Nakaner:} Sollte man das OSM-Wiki durch etwas Neues ersetzen? (Diskussion)
  \item \emph{snupo:} Wofür mappen wir? Für den Renderer? Router? Menschen? Maschinen? ein bisserl von allem?
\end{itemize}
\vfill
\justifying

\newpage
% neuer Layer für Geländeplan
%\DeclareNewLayer[background, oddorevenpage,  width=125mm,%
%height=169mm, contents={%
%  \includegraphics{wallpaper/raumplan-a6.pdf}%
%}]{raumplana6}
%\newpairofpagestyles[scrheadings]{raumplan}{}
%\AddLayersAtBeginOfPageStyle{raumplan}{raumplana6}
%\pagestyle{raumplan}
%\null
%\label{raumplan-page}
